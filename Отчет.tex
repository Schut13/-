\documentclass[12pt,a4paper]{scrartcl} 
\usepackage[utf8]{inputenc}
\usepackage[english,russian]{babel}
\usepackage{indentfirst}
\usepackage{misccorr}
\usepackage{graphicx}
\usepackage{amsmath}
\begin{document}
	\begin{titlepage}
		\begin{center}
			\large
			МИНИСТЕРСТВО НАУКИ И ВЫСШЕГО ОБРАЗОВАНИЯ РОССИЙСКОЙ ФЕДЕРАЦИИ
			
			Федеральное государственное бюджетное образовательное учреждение высшего образования
			
			\textbf{АДЫГЕЙСКИЙ ГОСУДАРСТВЕННЫЙ УНИВЕРСИТЕТ}
			\vspace{0.25cm}
			
			Инженерно-физический факультет
			
			Кафедра управления и информатики в технических системах
			\vfill
			
			\vfill
			
			\textsc{Отчет по практике}\\[5mm]
			
			{\LARGE Программаная реализация численного метода \textit{Интерполировать функцию, используя многочлен Лагранжа 
		}}
			\bigskip
			
			1 курс, группа 1ИВТ
		\end{center}
		\vfill
		
		\newlength{\ML}
		\settowidth{\ML}{«\underline{\hspace{0.7cm}}» \underline{\hspace{2cm}}}
		\hfill\begin{minipage}{0.5\textwidth}
			Выполнил:\\
			\underline{\hspace{\ML}} И.\,А.~Шатский\\
			«\underline{\hspace{0.7cm}}» \underline{\hspace{2cm}} 2022 г.
		\end{minipage}%
		\bigskip
		
		\hfill\begin{minipage}{0.5\textwidth}
			Руководитель:\\
			\underline{\hspace{\ML}} С.\,В.~Теплоухов\\
			«\underline{\hspace{0.7cm}}» \underline{\hspace{2cm}} 2022 г.
		\end{minipage}%
		\vfill
		
		\begin{center}
			Майкоп, 2022 г.
		\end{center}
	\end{titlepage}

	\section{Введение}
	\label{sec:intro}
	
	% Что должно быть во введении
	\begin{enumerate}
		\item Текстовая формулировка задачи
		\item Пример кода, решающего данную задачу
		\item График
		\item Скриншот программы
	\end{enumerate}
	
	Пример приведен в пункте ~\ref{sec:exp} на стр.~\pageref{sec:exp}.
	
	\section{Ход работы}
	\label{sec:exp}
	
	\subsection{Код приложения}
	\label{sec:exp:code}
	\begin{verbatim}
		public partial class Form1 : Form
    {
        public Form1()
        {
            InitializeComponent();
        }

        private void button1_Click(object sender, EventArgs e)
        {
            chart1.Series[0].Points.Clear();
            string Str = textBox1.Text.Trim();
            int NumPoints;
            bool isNum = int.TryParse(Str, out NumPoints);
            if (!isNum)
            {
                return;
            }
            var xValues = new double[NumPoints];
            var yValues = new double[NumPoints];
            var Rand = new Random();
            for (int i = 0; i < NumPoints; i++)
            {
                xValues[i] = i;
                yValues[i] = Rand.Next(NumPoints+1);
            }
            for (int i = 0; i < NumPoints; i++)
            {
                chart1.Series[0].Points.AddXY(xValues[i],
                LagrangeInterpolation(xValues, yValues, i));
            }
        }
  public static double LagrangeInterpolation(double[] x, double[] y, double xval)
        {
           double yval = 0.0;
           double Products = y[0];
           for (int i = 0; i < x.Length; i++)
            {
               Products = y[i];
            for (int j = 0; j < x.Length; j++)
                {
                    if (i != j)
                    {
                        Products *= (xval - x[j]) / (x[i] - x[j]);
                    }
                }
                yval += Products;
            }
            return yval;
        }
    }
	    }
	    
	\end{verbatim}
	
	\subsection{Пример формулы}
	\label{sec:mathexample}
	
	Интерполяционный многочлен Лагранжа:
	\begin{equation}\label{eq:solv}
	$L(x) = \sum_{i=0}^n (f(a_i) \prod_{j\neq i,j=0}^n \frac{x-a_j}{a_i-a_j} )$
		
	
\section{Пример вставки изображения}
\begin{figure}[h]
	\centering
	\includegraphics[width=0.4\textwidth]{142.png}
	\caption{Работа программы}\label{fig:par}
\end{figure}
	
	\section{Пример библиографических ссылок}
    http://geo.phys.spbu.ru/LDUS/files/books/LaTeX/LaTeX-Lvovsky.pdf
	
	\begin{thebibliography}{9}
		\bibitem{Knuth-2003}Кнут Д.Э. Всё про \TeX. \newblock --- Москва: Изд. Вильямс, 2003 г. 550~с.
		\bibitem{Lvovsky-2003}Львовский С.М. Набор и верстка в системе \LaTeX{}. \newblock --- 3-е издание, исправленное и дополненное, 2003 г.
		\bibitem{Voroncov-2005}Воронцов К.В. \LaTeX{} в примерах. 2005 г.
	\end{thebibliography}
\end{document}